% LaTeX Curriculum Vitae Template
%
% Copyright (C) 2004-2008 Jason Blevins <jrblevin@sdf.lonestar.org>
% http://jblevins.org/projects/cv-template
%
% You may use use this document as a template to create your own CV
% and you may redistribute the source code freely. No attribution is
% required in any resulting documents. I do ask that you please leave
% this notice and the above URL in the source code if you choose to
% redistribute this file.


\documentclass[a4paper]{article}


\usepackage{hyperref}
\usepackage{geometry}
\usepackage{lmodern}
\usepackage[utf8]{inputenc}
\usepackage[osf]{mathpazo}

\usepackage[danish]{babel}

\usepackage{lastpage}

\usepackage{fancyhdr}
\pagestyle{fancy}
\renewcommand{\headrulewidth}{0pt}
\renewcommand{\footrulewidth}{0pt}
\cfoot{\thepage\ af \pageref{LastPage}}

\hypersetup{
  colorlinks,
  urlcolor = black,
  pdfauthor={Anders Lemke-Holstein},
  pdfkeywords={},
  pdftitle={Curriculum Vitae},
  pdfsubject={Curriculum Vitae},
  pdfpagemode=UseNone
}

\geometry{textheight=9in, textwidth=6in}

% Make lists without bullets
\renewenvironment{itemize}{
  \begin{list}{}
    { \setlength{\itemsep}{5pt}
      \setlength{\parsep}{0pt}
      \setlength{\topsep}{0pt}
      \setlength{\leftmargin}{0em} } }{
  \end{list}}

% Customize section headings
\usepackage{sectsty}
\sectionfont{\sffamily}

\newcommand{\keywords}[1]{\small\textbf{Nøgleord:} \emph{#1}\normalsize} 

% Don't indent paragraphs.
\setlength\parindent{0em}
\setlength\parskip{0.5em}

% Make a nicer C++
\def\Cplusplus{C{\raise.5ex\hbox{\footnotesize ++ }}}

\begin{document}

{\huge\bf Anders Lemke-Holstein}

\bigskip
Præstegårds Allé 50 \\
2700 Brønshøj

\medskip

Telefon: 20 44 21 27

\medskip

Email: \href{mailto:anders@lemke.dk}{\tt anders@lemke.dk} \\
URL: \href{https://www.anderslemke.dk}{\tt www.anderslemke.dk} \\
GitHub: \href{https://www.github.com/anderslemke}{\tt github.com/anderslemke} \\
LinkedIn: \href{https://www.linkedin.com/in/anderslemke}{\tt linkedin.com/in/anderslemke} \\
Twitter: \href{https://www.twitter.com/anderslemke}{\tt twitter.com/anderslemke} 

\section*{Uddannelse}

Civilingeniør, cand.polyt, Informatik, Danmarks Tekniske Universitet, august 2008.

\section*{Nuværende}

\subsection*{Platform Architect and Lead Developer - Zetland - 2015}
\keywords{Event Sourcing, CQRS, Ruby on Rails, React, RabbitMQ, React Native}

Zetland er et dansksproget medie som udgiver kvalitetsjournalistik til over 23.000 medlemmer på deres egen prisvindende digitale platform.

Platformen består af et CMS, et API, en React frontend, og en React Native app (udgivet på Android og iOS).

Mine ansvarsområder indbefatter platformens arkitektur, implementering, integration, drift samt skalering. Herudover har jeg det tekniske og udviklende ledelsesansvar for teamet.

Zetland har solgt en licens til platformen til svenske Vi Media AB. De udgiver dermed deres journalistik digitalt på \href{https://vi.se}{\tt vi.se} på en tilrettet version af samme platform.

Besøg Zetland på \href{https://www.zetland.dk}{\tt www.zetland.dk}.

\subsection*{Stifter af Promise}
\keywords{Event Sourcing, CQRS, Ruby on Rails, Security, Encryption}

Promise er en utopisk drøm om, at have ét centralt sted at logge ind på alle online services. Så en central single sign-on platform.

Læs mere om Promise på \href{https://promiseauthentication.org}{\tt promiseauthentication.org} \\

\subsection*{hurtigfaktura.dk}


\section*{Tidligere ansættelser}

\subsection*{Developer - Firmafon - 2013-2015}
\keywords{Angular.js, Cordova, Ruby on Rails, Freeswitch, Puppet, RabbitMQ}

Som én af tre udviklere med primært ansvar for diverse apps. Både i browser og på mobiler.

Har desuden bygget intern sælgerafregning.

\subsection*{Senior Developer - Issuu - 2012-2013}
\keywords{Javascript, Ruby on Rails, node.js, SASS, Grunt}

Lead developer på \href{http://www.magmahq.com}{Magma} som er skrevet i Ruby on Rails.

Arbejder desuden på Issuu's frontend, hvor jeg har ansvar for udvalgte dele.

\subsection*{autobutler.dk - 2011-2012}
\keywords{Projektstyring, Ruby on Rails, Heroku}

Fungerende CTO med ansvar for videreudvikling og design af både kunderettede systemer samt interne systemer. Har koordineret med diverse freelancere.

Var ansvarlig for at tage applikationen fra prototype til færdigt produkt. Herunder fuld launch i Sverige. 

\subsection*{Benjamin Media - 2009-2010}

\keywords{FreeBSD, Ruby on Rails, Capistrano, Unicorn, PostgreSQL, serverarkitektur, Dankort, sikkerhed}

Hos Benjamin Media stod jeg primært for omstrukturering af serverarkitektur således at deres webbaserede applikationer kørte på en platform som kunne skalere.

Desuden stod jeg for udvikling af betaling for abonnementer med Dankort via web og integrering med moderselskabet Bonnier's abonnementssystem.

\subsection*{Gigahost ApS - 2008}

\keywords{Linux, PHP, webudvikling, Apache, MySQL, serveradministration, selvstændigt, Mogile FS}

Ansat ved Gigahost som civilingeniør på midlertidig basis.

Gigahost sælger webhoteller, primært til private, og benytter open source systemer til tekniske løsninger.

Konkret har jeg bl.a. sat det distribuerede filsystem Mogile FS op, implementeret en ny cron job runner, implementeret et system til automatisk fornyelse af domæner, samt konfigureret nameservere (PowerDNS) og implementeret administrationssystem hertil.

Desuden har jeg løst mindre opgaver relateret til serveradministration og vedligehold, og har fået godt kendskab til domæneadministration og DNS.
\bigskip
% Footer
\begin{center}
\begin{footnotesize}
Sidst opdateret: \today \\
\href{http://files.anderslemke.dk/cv.pdf}{\tt http://files.anderslemke.dk/cv.pdf}
\end{footnotesize}
\end{center}
\end{document}
